\newglossaryentry{glos:rechenaufwand}{name=Rechenaufwand, description={ beschreibt die Komplexität eines Verfahrens. Die Anzahl der Schritte, die für die Berechnung benötigt werden, dient als Kennzahl}} 

\newglossaryentry{glos:trainingsdatensatz}{name=Trainingsdatensatz, description={liefert die Datenbasis für die Schätzung der Modellparameter}} 

\newglossaryentry{glos:datamining}{name=Datamining,description={Systematische Anwendung statistischer Methoden auf einen grossen Datenbestand mit dem Ziel, neue Muster zu erkennen}}

\newglossaryentry{glos:praediktorvariable}{ plural={Prädiktorvariablen},name=Prädiktorvariable,description={Unabhängige Variable, die einen zu bestimmenden Einfluss auf die Kriteriumsvariable ausübt}}

\newglossaryentry{glos:kriteriumsvariable}{name=Kriteriumsvariable,description={Erklärte Variable, welche eine Wirkung misst}}

\newglossaryentry{glos:kreuzvalidierung}{name=Kreuzvalidierung,description={ bezeichnet Verfahren, bei denen die Vorhersagezuverlässigkeit eines Modells anhand von unabhängigen Teilstichproben bestimmt wird}}

\newglossaryentry{glos:exhaustive Verfahren}{name=exhaustive Verfahren,plural={exhaustiven Verfahren},description={ rechnen alle möglichen Modelle anhand der potentiellen Prädiktoren durch}}

\newglossaryentry{glos:schrittweise Verfahren}{name=schrittweise Verfahren,description={rechnen Modelle durch  schrittweise Hinzunahme beziehungsweise Weglassen potentieller Prädiktoren}}

\newglossaryentry{glos:Modellguete}{name=Modellgüte,description={beschreibt, wie gut ein Modell gegebene Daten vorhersagen kann}}

\newglossaryentry{glos:Overfitting}{name=Overfitting,description={beschreibt eine mangelnde Generalisierbarkeit aufgrund eines Modells, das zu sehr an die Trainingsstichprobe angepasst ist}}

\newglossaryentry{glos:Multikollinearitaet}{name=Multikollinearität,description={ ist ein Problem der Regressionsanalyse und liegt vor, wenn zwei oder mehr Prädiktoren stark miteinander korrelieren}}

\newglossaryentry{glos:Stabilitaet}{name=Stabilität,description={beschreibt, wie stark die Vorhersagen eines Modells im Generellen variieren}}

\newglossaryentry{glos:Maximum-Likelihood-Methode}{name=Maximum-Likelihood-Methode,description={ wird benützt, um die unbekannten Parameter einer Funktion aus gemessenen Daten zu bestimmen}}
