\documentclass[12pt,doc]{apa}
\usepackage{blindtext}
\usepackage[a4paper, left=25mm, right=40mm]{geometry}
\usepackage[]{natbib}

\setlength{\parskip}{1em} % 1ex plus 0.5ex minus 0.2ex}
\setlength{\parindent}{0pt}

\usepackage[utf8]{inputenc}
\usepackage[english]{babel}

\begin{document}

\title{Predictor selection in regression models}
\shorttitle{Predictor selection in regression models} 
\author{Literaturarbeit vorgelegt von Markus Graf}
\date{\today}
\affiliation{Department of Psychology University of Zürich}
\abstract{In vielen psychologischen Bereichen geht es darum, Kriteriumsvariablen durch Prädiktorvariablen möglichst gut vorherzusagen. Wenn viele potentielle Prädiktorvariablen in Frage kommen und es keine theoretischen Gründe gibt, die nur ganz bestimmte Prädiktorvariablen nahelegen, werden in Anwendungssituationen oft automatische Verfahren der Auswahl von Prädiktorvariablen verwendet, um mit möglichst wenigen Prädiktoren eine möglichst gute Vorhersage des Kriteriums zu erreichen, beispielsweise die sog. "Stepwise"-Methode in multiplen Regressionsmodellen. Die Literaturarbeit soll einen Überblick über verschiedene existierende Möglichkeiten zur Selektion von Prädiktorvariablen in Regressionsmodellen geben, und deren Eignung für psychlogische Anwendungen kritisch diskutieren.}
\maketitle
\setlength{\parindent}{0pt}
\newpage
\tableofcontents
\newpage
\section{Planung / Brainstorming}
\begin{enumerate}
\item Was ist ein multivariates Regressionsmodel
\item Zwei beispielhafte psychologische Studien, wobei eine klare theoriegeleitet Prädiktoren beinhaltet. Die Andere sollte viele prädiktoren beihnalten deren Rolle nicht klar auf der Hand liegen.
\item Anhand des zweiten Beispiels soll aufgezeigt werden, warum automatische selektion von Prädiktoren Sinn machen kann.
\item Dabei sollen verschiedene Fälle aufzeigen wann dies in der psychologischen Forschung Sinn macht.
\item Anhand der beschriebenen Fälle soll aufgezeigt werden welche Methoden zur automatischen Selektion existieren und wo die Limitierungen liegen.
\item Abschliessend sollen die wichtigsten Methoden für die psycholoie Zusammengefasst werden.
\end{enumerate}
\section{Introduction}
The estimated relation between several predictor variables and a dependent variable is called multiple correlation where  the multiple regression equation is used to predict a dependent variable due to one or more predictor variable. \cite[Chap 13]{bortz2011}.

\blindtext
\section{Section A}
\blindtext
\section{Section B}is used to 
\blindtext
\section{Section C}
\blindtext
\section{Discussion}
\blindtext
\section{Supplements}
\newpage
\bibliographystyle{apacite} 
\bibliography{literature}
\section*{Statement of authorship}
I declare that this work titled ``\title'' has been composed by myself, and describes my own work, unless otherwise acknowledged in the 
text.  

If the paper has been authored by more than one person, I confirm that all parts of the paper have been clearly assigned to the respective author.

This work has not been and will not be submitted  for any other degree or the obtaining of ECTS points at the University of Zurich or any other institution of higher education. 

All sentences or passages quoted in this paper  from other people's work have been specifically acknowledged by clear cross-referencing to author, work and page(s). Any illustrations which are not the work of the author have been used with the  explicit permission of the originator and are specifically acknowledged.  

I understand that failure to specifically acknowledge all used work amounts to plagiarism and will be considered grounds for failure and will have judicial and disciplinary consequences according §7ff of the ``Disziplinarordnung der Universität Zürich'' as well as § 36 of the ``Rahmenordnung für das Studium in den Bachelor- und Master-Studiengängen der Philosophischen Fakultät der Universität Zürich''. 

With my signature I declare the accuracy of these specifications.
\\
\\
Name: Markus Graf\\
Matriculation number:  08-91271-9\\
\\
\\
\\
\\
\\
\\
............................................................................\\
Zürich, \today


\end{document}
