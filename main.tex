\documentclass[english,12pt,doc]{apa}
\usepackage[a4paper, left=25mm, right=40mm]{geometry}

\usepackage{apacite}
\usepackage{tikz}
\usepackage[german]{babel}
\usepackage{verbatim}
\usepackage{setspace}
\usepackage{graphicx}
\usepackage{caption}
\usepackage{prettyref}
\usepackage{titleref}

\onehalfspacing
\setlength{\parskip}{1em} % 1ex plus 0.5ex minus 0.2ex}
\setlength{\parindent}{0pt}

\usepackage[utf8]{inputenc}

\usepackage[]{blindtext}
\rightheader{Automatische Verfahren zur Prädiktorauswahl}

\begin{document}

\title{Automatische Verfahren zur Prädiktorauswahl in Regressionsmodellen}
\shorttitle{Prädiktorauswahl in Regressionsmodellen} 
\author{Literaturarbeit vorgelegt von \\ Markus Graf (markus.graf@uzh.ch)}
\date{\today}
\affiliation{am  Psychologisches Institut der Universität Zürich\\ Betreut durch Dr. Christina Werner}
\abstract{Eigener Abstract: - 

Themenvorgabe: In vielen psychologischen Bereichen geht es darum, Kriteriumsvariablen durch Prädiktorvariablen möglichst gut vorherzusagen. Wenn viele potentielle Prädiktorvariablen in Frage kommen und es keine theoretischen Gründe gibt, die nur ganz bestimmte Prädiktorvariablen nahelegen, werden in Anwendungssituationen oft automatische Verfahren der Auswahl von Prädiktorvariablen verwendet, um mit möglichst wenigen Prädiktoren eine möglichst gute Vorhersage des Kriteriums zu erreichen, beispielsweise die sog. ``Stepwise''-Methode in multiplen Regressionsmodellen. Die Literaturarbeit soll einen Überblick über verschiedene existierende Möglichkeiten zur Selektion von Prädiktorvariablen in Regressionsmodellen geben, und deren Eignung für psychlogische Anwendungen kritisch diskutieren.}
\maketitle
\setlength{\parindent}{0pt}
\newpage
\tableofcontents
\newpage
\section{Einführung}
Das Standardverfahren um den quantitative Zusammenhang zwischen einer abhängigen und einer unnabhängigen Variablen zu beschreiben stellt die Regressionsanalyse dar. Begründet wurde dieses Verfahren durch Carl Friedrich Gauss in seiner Schrift, in der er, mit Hilfe der Methode der kleinsten Quadrate, die Bewegung der Himmelskörper um die Sonne im Kegelschnitt beschrieb \cite{gauss1809theoria}. 

Im Unterschied zur einfachen linearen Regression, werden in einem multiplen Regressionsmodel mehrere unabhängige Variablen mit einbezogen. Es resultiert eine Regressionsgleichung welche zur Vorhersage einer Kriteriumsvariable aufgrund mehrerer Prädikatorvariablen genutzt wird  \cite[S. 448]{bortz2011}. 
Die Gretchenfrage stellt sich nun welche Prädikatoren nun ein Model am besten erklären. Zu beginn der Psychologischen Forschung mussten Modelle von Hand berechnet werden. Zwangsläufig wurden wenige Prädikatoren erhoben und einfache modelle Gerechnet. Friedman analysierte beispielsweise 1944 die Langlebigkeit von Turbinenschaufeln in Abhängikeit von Stress, Temperatur und einigen Legierungsparametern. Zwar wurde die Berechnung nicht mehr von Hand durchgeführt, doch benötigte eine Regressionsschätzung inklusive Berechnung der Teststatistiken rund 40 Stunden \cite[p.2]{armstrong2011illusions}. Jeder durchschnittliche Computer erledigt dies heutzutage Sekundenbruchteile. 

Mit diesem technische Fortschritt einhergehend wurden Verfahren entwickelt, welche möglichen Kombinationen von Prädiktoren berücksichtigen und gegeneinander testen. In einem ersten Teil dieser Arbeit werden die wichtigsten Verfahren dargestellt. 

Es können beliebig viele potentiel erklärende Variablen erhoben werden um sich komplexe Model generieren zu lassen. Menschen tendieren zu glauben, dass komplexe Probleme komplexe Lösungen benötigen. Die Forschung zeigt jedoch, dass gerade das Umgekehrte der Fall ist \cite[p.3]{armstrong2011illusions}. sud
Insbesondere Gigerenzer demonstrierte eindrucksvoll wie mit einfachen Rekognitionsheuristiken bessere Vorhersagen gemacht werden konnten als mit komplexen statistischen Modellen \cite{borges1999can}. Komplexe Modelle können sehr gute Vorhersagen liefern innerhalb des Referenzdatensatzes, doch oft scheitern die Vorhersage beim Versuch, diese zu generalisieren. Der zweite Teil befasst sich mit dem Problem der Überanpassung komplexer Modelle und diskutiert mögliche Lösungsansätze.

\section{Automatische Modelwahlverfahren}
In der psychologischen Forschung kommen oft Ex-post-facto-Designs zum Einsatz. Dies insbesondere weil viele Daten mit geringem Aufwand erhoben werden können. Oft werden  Daten gleich für mehrere Studien erhoben und in anderen Studien verwendet. Daraus resultieren viele potenzielle Prädikatorvariablen, oft ohne das es theoretische Gründe gibt nach denen die Auswahl einzuschränken ist, um das ``beste'' Modell berechnen. Je nach Verfahren, können verschiedene Kennwerte zur interpretation der Modellgüte herangezogen werden, welche nun kur vorgestellt werden.

\subsection{Exhaustiv Regression} 
Eine naive Herangehensweise ist alle möglichen Modelle welche mit $p$ Prädiktoren möglich sind durchzurechnen. Zur Beurteilung der Modellgüte kann die mittlere quadratische Abweichung ($S_p$-Kriterium) herangezogen werden. Das ``beste'' Modell hat die kleinste mittlere quadratische Abweichung. \citeA[p.6]{thompson1978selection} sieht einzig den Nachteil darin, dass der Rechenaufwand exponentiell mit der Anzahl zu berücksichtigender Prädikatoren steigt. Es müssen immer $2^p-1$ Modelle berechnet werden, bei 5 Prädikatoren sind dies 31 Modelle, bei 10 bereits 1023 usw. Während früher eingeschränkte Rechenkapazität oft ein ökonomischer Faktor war - es musste Rechenzeit in einem Rechenzentrum reserviert werden, spielt die Rechengeschwindigkeit auf modernen Systemen eine untergeordnete Rolle, da insbesondere in der Psychologie oft nur eine Handvoll Prädikatoren durchgerechnet werden müssen.

\subsection{Schrittweise Regression} Das optimale Model beinhaltet jeden Prädikator, der die Voraussage bezüglich des getesteten Datensatzes auch nur minimal verbesssert. Es stellt sich die Frage ob diese minimale Verbesserung noch nützlich ist. Die ``stepwise''-Verfahren arbeiten wesentlich liberaler, in dem Prädikatoren hinzugefügt oder eliminiert werden, je nach deren Relevanz für die Modellgüte. Es werden Abbruchkritereien festgelegt, nach welchen das Modell als angemessen zu betrachten sind. Dies hat gegenüber der exhaustiven Verfahren den Vorteil, dass deutlich nicht alle Modelle berechnet werden müssen und entsprechend schneller Lösungen gefunden werden.

Innerhalb der schrittweisen Verfahren unterscheidet man zwischen \textit{forward selection} und \textit{backward elimination}. Ausgehend vom leeren Modell werden in der ersten Variante schrittweise weitere Variable der Nützlichkeit nach in das Modell integriert, bis eine Abbruchbedingung erfüllt ist. 
\begin{figure}[hb]
	\centering
	\includegraphics[width=\textwidth]{forward_stepwise.png}
	\caption{Forward Selection}
	\label{fig:forward_stepwise}
\end{figure}

In der zweiten Variante werden alle Prädikatoren in das Modell integriert und schlechte entfernt, wiederum bis das Abbruchkriterium erreicht ist. 
\begin{figure}[hb]
	\centering
	\includegraphics[width=\textwidth]{backward_stepwise.png}
	\caption{Backward Elimination}
	\label{fig:backward_stepwise}
\end{figure}

Die Aufnahme einer neuen Variable kann dazu führen, dass eine bereits im Model verhandene Variable obsolet wird. Um diesem Umstand Rechnung zu tragen werden oft forward selection und backward elimination kombiniert \cite[p. 461]{bortz2011}. 

In selltenen Fällen kann es vorkommen, dass zwei Variablen für sich das Kriterium in die Regressionsgleichung aufgenommen zu werden nicht erfüllen, jedoch zusammen zum Vorhersagepotetial einen substantiellen Beitrag leisten \cite[p.261]{jacob2003applied}. Schrittweise Verfahren sind entsprechend nicht in der Lage solche Effekte mit zu berücksichtigen. 

Entgegen dem exhaustiven Verfahren besteht bei schrittweisen Verfahren das Problem, dass unter Umständen nicht das optimale Modell gefunden wird. Da die Nützlichkeitsunterschiede, welche oft nur geringe statistische Bedeutung haben, das Modell bestimmen, bezeichnet \citeA[p. 462]{bortz2011} dieses Verfahren eher zu den explorativen gehörend. \citeA[p. 56ff]{harrell2001regression} lehnt das Verfahren gar ab und führt ins Feld, dass sämtliche statistischen Prinzipien verletzt würden. \citeA{berk1978comparing} zeigte jedoch in einem Vergleich, dass die durchschnittliche Differenz der Fehlerquadratsummen zwischen exhaustiven und schrittweriser Regression kaum 7\% übertrifft. 

Zentrales Element der schrittweisen Regression ist das Mass zur Beurteilung der  Modellanpassung, welche besagt, weshalb und wann ein Model als akzeptabel zu betrachten ist. Als Folge dessen wird damit auch die Anzahl relevanter Prädiktoren bestimmt.

\paragraph{Bestimmtheitsmass}  Das Quadrat des multiplen Korrelationskoeffizienten $R^2$ besagt wieviel systematische Varianz aufgeklärt wird. Je grösser die Zahl der unabhängigen Variablen ist, desto grösser wird das Bestimmtheitsmass, weshalb eine korrektur vorgenommen werden muss. Insbesondere bei kleinem $R^2$ sollte auf Signifikanz getestet werden, entsprechend wird in schrittweisen Verfahren Prädikatoren nicht einzig aufgrund von $R^2$ selektiert. 

\paragraph{Signifikanztest} Das Verfahren wird beendet, wenn kein Pädikator mehr vorhanden ist, der das Vorhsagepotintal signifikant erhöht. Das vergleichen zweier Regressionsgleichungen mittels Signifikanztests bedingt, dass diese geschachtelt sein müssen, das kleinere Modell muss im grösseren enthalten sein \cite[p. 508]{jacob2003applied}.

\citeA[p. 269]{derksen2011backward} diskutieren mehrere Empfehlungen für Signifikanzniveaus und weisen darauf hin, das sich über mehrere Tests der $\alpha$-Fehler kummuliert. In  Simulationen mit artifiziellen Daten zeigen \citeA{mundry2009stepwise} das  Problem multipler Tests beispielhaft auf. Daraus resultiertend lehnen sie die Verwendung der schrittweisen Regression gar ab.
 
\paragraph{Informationskriterium (AIC / BIC)} Das Akaikes Informationskriterium (AIC) und Bayessche Informationskriterium (BIC) basieren auf der Maximum-Likelihood-Methode, müssen nicht geschachtelt sein und berücksichtigen die Komplexität des Models anhand der Prädiktoranzahl. Die beiden Kennwerte strafen dem Prinzip der Sparsamkeit entsprechend komplexität. \ref{xx}


 geht mit dem schrittweisen Verfahren hart ins Gericht. Es sei keine statistische Methode, da es  jedes Prinzip des statistischen Schätzens und der Hypothesentestung verletze. Dabei führt er ins Feld, dass die F und $\chi^2$ Teststatistik nicht über die geforderte Verteilung verfüge und vorgesehen sind vordefinierte Hypothesen zu teste. Die Auswahl der berücksichtigt die Kolinearität nicht, welche die Auswahl willkürlich machen lässt.

\section{Zusammenfassung / Diskussion}
\label{xx}
\blindtext

\newpage
\bibliographystyle{apacite} 
\bibliography{literature}
\newpage
\section{Anhang}
\blindtext
\newpage
\section{Selbstständigkeitserklärung}
\blindtext
\begin{comment}
I declare that this work titled ``\title'' has been composed by myself, and describes my own work, unless otherwise acknowledged in the 
text.  

If the paper has been authored by more than one person, I confirm that all parts of the paper have been clearly assigned to the respective author.

This work has not been and will not be submitted  for any other degree or the obtaining of ECTS points at the University of Zurich or any other institution of higher education. 

All sentences or passages quoted in this paper  from other people's work have been specifically acknowledged by clear cross-referencing to author, work and page(s). Any illustrations which are not the work of the author have been used with the  explicit permission of the originator and are specifically acknowledged.  

I understand that failure to specifically acknowledge all used work amounts to plagiarism and will be considered grounds for failure and will have judicial and disciplinary consequences according §7ff of the ``Disziplinarordnung der Universität Zürich'' as well as § 36 of the ``Rahmenordnung für das Studium in den Bachelor- und Master-Studiengängen der Philosophischen Fakultät der Universität Zürich''. 

With my signature I declare the accuracy of these specifications.
\\
\\
Name: Markus Graf\\
Matriculation number:  08-91271-9\\
\\
\\
\\
\\
\\
\\
............................................................................\\
Zürich, \today
\end{comment}


\end{document}
