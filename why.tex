\section{Sinn und Zweck automatisierter Modellwahl}
In psychologischen Fragestellungen kommt es vor, dass viele Prädiktoren in ein Modell einfliessen oder potentiell für ein Modell in Frage kommen.
Die Frage, welche Prädiktoren nun ein Modell am besten beschreiben ist dabei die eigentliche Gretchenfrage. 

Unterteilen lässt sich die automatisierte Modellwahl in (a) eine explorative, und (b) eine optimierende Anwendung. 
Im Falle der explorativen Anwendung fehlen grösstenteils theoretische Begründungen für die Auswahl bestimmter Prädiktoren. Ein Beispiel für eine solche Anwendung liefert eine Studie, die Prädiktoren des Alltagstransfers eines stationär erlernten Entspannungstrainings suchte \cite{023755520080101}.
Oft werden Daten gleich für mehrere Studien erhoben und in anderen Studien verwendet.
Diese systematische Anwendung automatischer Modellwahlverfahren, mit dem Ziel neue Muster zu erkennen, ist damit eine Aufgabenstellung des Datamining. 
Die so gewonnen Daten können zu neuen Fragestellungen führen. 

Der zweite Anwendungsfall ergibt sich, wenn bereits ein Modell vorhanden ist. Insbesondere komplexe Modelle sind meist schlecht generalisierbar. Automatisierte Verfahren können  helfen, Prädiktoren zu erkennen, welche die Komplexität unnötig erhöhen.